\documentclass[12pt, a4paper]{article}
\PassOptionsToPackage{sharp}{prettytex/boxes}
\usepackage{prettytex/base}

\setlength{\topmargin}{0.0in}
\setlength{\oddsidemargin}{0.33in}
\setlength{\textheight}{9.0in}
\setlength{\textwidth}{6.0in}
\renewcommand{\baselinestretch}{1.25}

\usepackage{prettytex/math}
\usepackage[nameinlink]{cleveref}
\usepackage[cleveref]{prettytex/math-theorems}
\usepackage{prettytex/mathematicians}
% \usepackage{prettytex/gfx}
% \usepackage{prettytex/code}
% \usepackage{prettytex/pseudo}
\usepackage{prettytex/thesis}
\usepackage{bbm}

\setlength{\headheight}{19.53pt}
\setlength{\headsep}{1.8em}
\setlength{\belowcaptionskip}{-12pt}
\addbibresource{sources.bib}
% \setminted{fontsize=\footnotesize}
\AfterEndEnvironment{minted}{\vspace*{-0.8cm}}
% \tikzexternalize[prefix=tikz/]
\renewcommand{\operatorcolor}{black}

\newcommand{\chebyshev}{Chebyshev\xspace}
\newcommand{\tschebfun}{\textcolor{themecolor3}{TschebFun}\xspace}
\newcommand{\heatfun}{\textcolor{themecolor3}{HeatFun}\xspace}
\newcommand{\identity}{\mathbbm{1}}

\newcommand{\topictitle}{Unsupervised Semantic Field Analysis \\ \large by example of the Heat Equation}
\newcommand{\candidatenumber}{1072462}
\newcommand{\course}{Networks}

\title{\topictitle}
\author{Candidate \candidatenumber}
\date{\today}

% ✅ Prepare report

\begin{document}
  \pagestyle{plain}
  \begin{center}
    \vspace*{-2.5cm}
    \Large \topictitle \\
    \vspace{.3cm}

    \normalsize Special Topic on \textcolor{themecolor3}{\textsc{\course}}\\
    \normalsize Candidate Number: \textcolor{themecolor3}{\candidatenumber}
    \vspace{.3cm}
  \end{center}

  \begin{abstract}
    \label{abstract}
    This work will attempt to
  \end{abstract}

  % \begin{figure}[H]
  %   \centering
  %   \includegraphics[width=\linewidth]{figures/screenshot.png}
  %   \caption{Screenshot of the graphical user interface. After entering an initial expression $u_0(x)$, depicted in grey, the simulation will run upon pressing 'Start'. The solution at time $t$, depicted in blue, is represented as a \chebyshev series of degree 29.}
  % \end{figure}

  \pagebreak
  \pagestyle{normal}

  % \tableofcontents
  % \pagebreak

  \section{Motivation}
  For non-periodic problem settings, \chebyshev series are a fantastic choice \parencite{cw-biemann}.

  \pagebreak
  \section{Introduction}
  Let $\N$ denote the positive integers, so $0 \notin \N$.

  \begin{definition}{Undirected Graph}{undirected-graph}
    A graph $G = (V, E)$ with vertices $V$ and edges $E \subseteq V \times V$ is undirected if and only if $(v_i, v_j) \in E \Rightarrow (v_j, v_i) \in E \quad \forall\; v_i, v_j \in V$.
  \end{definition}

  \begin{definition}{Adjacency Matrix}{adjacency-matrix}
    Let $A \in \{0, 1\}^{n \times n}$ denote the symmetric adjacency matrix of an undirected graph $G = (V, E)$. Its entries are given by $a_{ij} = \{A\}_{ij} = \identity_{(v_i, v_j) \in E}$, so $a_{ij} = 1$ if vertex $v_i$ is connected to $v_j$ and $0$ otherwise. By construction, $A = A^T$.
  \end{definition}

  Further let $m := |E|$ and $n = |V|$ denote the number of edges and vertices, respectively.
  The degree $d_i$ of a vertex $v_i \in V$ is defined by the number of edges connecting to it, so $$d_i := \deg(v_i) = \big|\left\{(v_j, v_k) \in E \;|\; v_j = v_i\right\}\big|\,,$$ for an undirected graph $G = (V, E)$. By the Handshaking lemma, $$\sum_{i=1}^{n} d_i = \sum_{v \in V} \deg(v) = 2m\,.$$

  \begin{definition}{Graph Clustering}{clustering}
    Let $C = \left\{C_i \subseteq V \right\}_{i=1 ... n_C}$ denote a clustering of $G = (V, E)$ into $n_C \in \N$ clusters where $C_i \cap C_j = \{\} \; \forall\,i, j \in \{1, ..., n_C\}$ and $\bigcup_{i=1}^{n_C} C_i = V$. Let $s_i \in \{1, ..., n_C\}$ denote the assigned cluster of vertex $v_i \in V$.
  \end{definition}

  \begin{definition}{Modularity}{modularity}
    For a given undirected graph $G = (V, E)$ and clustering $C$, let $$Q := \frac{1}{2m} \sum_{i=1}^{n} \sum_{j=1}^{n} \left(A_{ij} - \frac{d_i d_j}{2m}\right) \delta(s_i, s_j)\,,$$ with $\delta(\cdot, \cdot)$ the Kronecker delta indicating whether two vertices $v_i$ and $v_j$ belong to the same cluster \parencite{grindrod-lecture-notes}.
  \end{definition}

  Modularity is a measure of the quality of a clustering (also referred to as a partitioning) of $G$. It can also be written as
  $$Q = \frac{1}{2m} \sum_{c=1}^{n_C} \left[\sum_{v_i, v_j \in C_i} \left(A_{ij} - \frac{d_i d_j}{2m}\right)\right]\,.$$

  \section{Clustering Algorithms}
  \subsection{Chinese Whispers}
  \cite{cw-biemann}.
  \subsection{Watset}
  \cite{watset}.
  \subsection{Fiedler}
  \cite{fortunato}.
  \subsection{Louvain}
  \cite{grindrod-lecture-notes}.

  \pagebreak
  \section{Discussion and Outlook}
  Bla

  \pagebreak
  \printbibliography

  \appendix
  \section{Appendix Things?}
\end{document}
