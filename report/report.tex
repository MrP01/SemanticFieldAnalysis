\documentclass[12pt, a4paper]{article}
\PassOptionsToPackage{sharp}{prettytex/boxes}
\usepackage{prettytex/base}

\setlength{\topmargin}{0.0in}
\setlength{\oddsidemargin}{0.33in}
\setlength{\textheight}{9.0in}
\setlength{\textwidth}{6.0in}
\renewcommand{\baselinestretch}{1.25}

\usepackage{prettytex/math}
\usepackage[nameinlink]{cleveref}
\usepackage[cleveref]{prettytex/math-theorems}
\usepackage{prettytex/mathematicians}
\usepackage{prettytex/gfx}
\usepackage{prettytex/code}
\usepackage{prettytex/pseudo}
\usepackage{prettytex/thesis}

\setlength{\headheight}{19.53pt}
\setlength{\headsep}{1.8em}
\setlength{\belowcaptionskip}{-12pt}
\addbibresource{sources.bib}
\setminted{fontsize=\footnotesize}
\AfterEndEnvironment{minted}{\vspace*{-0.8cm}}
\tikzexternalize[prefix=tikz/]
\renewcommand{\operatorcolor}{black}

\newcommand{\chebyshev}{Chebyshev\xspace}
\newcommand{\tschebfun}{\textcolor{themecolor3}{TschebFun}\xspace}
\newcommand{\heatfun}{\textcolor{themecolor3}{HeatFun}\xspace}

\newcommand{\topictitle}{Semantic Field Analysis \\ \large by example of the Heat Equation}
\newcommand{\candidatenumber}{1072462}
\newcommand{\course}{Networks}

\title{\topictitle}
\author{Candidate \candidatenumber}
\date{\today}

% ✅ Prepare report

\begin{document}
  \pagestyle{plain}
  \begin{center}
    \vspace*{-2.5cm}
    \Large \topictitle \\
    \vspace{.3cm}

    \normalsize Special Topic on \textcolor{themecolor3}{\textsc{\course}}\\
    \normalsize Candidate Number: \textcolor{themecolor3}{\candidatenumber}
    \vspace{.3cm}
  \end{center}

  \begin{abstract}
    \label{abstract}
    This work will attempt to
  \end{abstract}

  % \begin{figure}[H]
  %   \centering
  %   \includegraphics[width=\linewidth]{figures/screenshot.png}
  %   \caption{Screenshot of the graphical user interface. After entering an initial expression $u_0(x)$, depicted in grey, the simulation will run upon pressing 'Start'. The solution at time $t$, depicted in blue, is represented as a \chebyshev series of degree 29.}
  % \end{figure}

  \pagebreak
  \pagestyle{normal}

  % \tableofcontents
  % \pagebreak

  \section{Motivation}
  For non-periodic problem settings, \chebyshev series are a fantastic choice \parencite{cwBiemann}.

  \pagebreak
  \section{Here we go}
  Let $\N$ denote the nonnegative integers, so $0 \in \N$.

  \pagebreak
  \section{Discussion and Outlook}
  Bla

  \pagebreak
  \printbibliography

  \appendix
  \section{Appendix Things?}
\end{document}
